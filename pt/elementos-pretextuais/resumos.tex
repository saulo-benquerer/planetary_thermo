% ----------------------------------------------------------
% RESUMOS
% ----------------------------------------------------------

% resumo em português
\setlength{\absparsep}{18pt} % ajusta o espaçamento dos parágrafos do resumo
\begin{resumo}

A Crise Climática gera bastante angústia ao redor do mundo e, mesmo que a sociedade pense que a culpa é da queima de combustíveis fósseis, ainda é difícil pensar em uma sociedade que não utilize petróleo. No entendimento atual, o dióxido de carbono(CO2) é o principal culpado do aumento do Efeito Estufa que gera o Aquecimento Global. Essa tese é contrária a visão das Mudanças Climáticas atual e coloca os sistemas de Ar-Condicionados como culpados.
\par
Primeiro será explicado as razões para essa mudança de perspectiva e será proposto um experimento para confirmar ou negar essa tese, e em seguida, trabalhos futuros envolvendo o uso de um novo sistema de refrigeração

    \vspace{\onelineskip}
    \noindent 
    \textbf{Palavras-chaves}: Aquecimento Global, Crise Climática, Efeito Estufa, Ar-condicionados, Refrigeração Geotérmica
\end{resumo}

% resumo em inglês
\begin{resumo}[Abstract]
 \begin{otherlanguage*}{english}

The Climate Crisis generates a lot of distress around the world and even as the scientific consensus thinks that the use of fossil fuel is the culprit - it's still hard to imagine a world without the use of oil. The current understanding is that carbon dioxide (CO2) increases the Greenhouse Effect that in turn generates the Global Warming. This thesis opposes this current notion of the Climate Change and blames the air conditioning systems.
\par
The rationale behind this change of perspective will be explained in the beginning and then a new experiment to prove the hypothesis will be proposed. Finally, future work will be suggested involving a new refrigeration system.

    \vspace{\onelineskip}
    \noindent 
    \textbf{Key-words}: Global Warming, Climate Crisis ,Greenhouse Effect, Air Conditioning, Geothermal Refrigerator
 \end{otherlanguage*}
\end{resumo}
