\chapter[Conclusões]{Conclusões}

A tecnologia de refrigeração geotérmica associada a técnicas mais modernas de agricultura tem o potencial de melhorar o rendimento e qualidade da comida para combater a fome que ainda assola no mundo. E ainda por cima, aplicada nos entornos de regiões desertificadas, pode redirecionar a água proveniente do derretimento das calotas polares para o combate à desertificação. Em resumo, essa tese não apenas deve ser lida como uma esperança contra a Crise climática, mas como o início de uma Revolução Verde no planeta. Se os resultados do experimento aqui descrito forem afirmativos, seria o início de acordos climáticos globais que envolvem a luta contra a fome e contra a desertificação no mundo.

A queima de combustíveis fósseis são essenciais para o funcionamento da sociedade moderna e mesmo que atualmente são considerados a causa das catástofres climáticas, ainda é difícil pensar em uma sociedade sem o seu uso. Essa tese tem como principal objetivo colocar a culpa das mudanças climáticas onde deve, que são nos ar-condicionados, porém, os combustíveis fósseis ainda são uma fonte de energia não renovável e é necessário que a sociedade comece a se planejar para a \textit{Era Pós-Petróleo}.