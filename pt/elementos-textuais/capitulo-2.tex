\chapter[Efeito Estufa]{Efeito Estufa}

O efeito estufa é o processo de reabsorção dos raios infravermelhos pela atmosfera e é essencial para a vida, pois se este não existisse, o planeta seria muito mais frio: por volta de -18°C \cite{Nasa-cold}. O problema com a forma atual de estudar esse fenõmeno é que estamos tentando dissecar a contribuição dos diversos componentes da atmosfera e estudá-los separadamente.

O que chamamos de efeito estufa é a resistência gerada pela pressão radioativa solar em contraponto com a radiação refletida pela superfície do planeta. Sabemos que a radiação emitida por uma estrela decresce com o quadrado da distância. A temperatura máxima que a Terra pode atingir depende muito mais da intensidade de radiação solar recebida pela posição que ocupamos no cosmos do que com a composição da atmosfera em si. Pois a posição de um planeta em relação a sua estrela não apenas afeta a quantidade de energia recebida, mas também influencia a taxa em que este perde calor para o espaço sideral. Pois não existe vácuo perfeito então a temperatura do espaço sideral em volta de um planeta também é diretamente influenciado pela distância em que ele está dessa \cite{Cosmic-wave}.

Em outras palavras, apesar do efeito estufa ter grande influência no clima global, existe um limite para seu efeito de aquecimento. O Efeito Estufa é como um cobertor:  diminui a taxa de perda de calor, porém é incapaz de realmente gerar um aquecimento. Existe uma temperatura máxima que a Terra pode atingir que é influenciada pela posição da Terra em relação ao Sol e a intensidade da radiação enviada por este. Esses dois fatores mudam apenas em escalas de tempo cósmicas, sendo portanto possível estabilizar o clima do planeta à longo prazo.

Os pólos recebem uma menor incidência de raios solares e, devido ao gelo, refletem também uma parte deles,  potencializando o resfriamento do local. Com isso, existe um fluxo de calor constante entre o Equador e os pólos. A todo momento o planeta reflete parte do calor como energia luminosa para o espaço sideral. Por isso quando o Voyager tirou a sua última foto, o planeta pode ser visto como o "Pale Blue Dot", se a Terra não irradiasse energia a todo momento para o cosmos, nosso planeta não poderia ser visto (\autoref{blue-dot}).

\begin{figure}[ht]
    \centering
    \includegraphics[scale=0.25]{pictures/pale-blue-dot.jpg}
    \caption{Imagem da Terra retirada pelo Voyager 1}
    \label{blue-dot}
    \legend{Fonte: NASA/JPL-Caltech}
\end{figure}

Obviamente essa é uma visão simplificada da questão, pois o gelo intensifica o grau de reflexão dos raios solares e pode chegar até um ponto crítico em que gera as Eras do Gelo. Outra razão para não fazer sentido dissecar o grau de efeito estufa de cada componente atmosférico é por que dependendo da concentração o seu efeito pode inverter. Dias muito nublados e/ou com muita fumaça atmosférica faz com que mais raios solares sejam refletidos pela atmosfera do que reabsorvidos pelo Efeito Estufa, ou seja, os mesmos componentes podem afetar o clima de forma diferente em circunstâncias diversas.

Além disso, devido a rotação do planeta existe a diferença entre dia e noite em lados opostos da Terra, essa diferença também influencia os ventos terrestres. Pois durante a noite o planeta perde energia térmica de forma mais intensa para o cosmos, o que acaba trazendo ventos das áreas mais aquecidas para essas regiões devido ao fluxo natural do calor.

O mais famoso gás do efeito estufa é o dióxido de carbono, que é produzido principalmente pela queima de combustíveis fósseis tais como petróleo ou carvão. Porém, existe outros gases que possuem essa propriedade, tais como o metano - que gera bastante discussão em termos do efeito da agropecuária nas mudanças climáticas. 
As plantas absorvem dioxido de carbono para produzir matéria orgânica e muitas evoluiram no período Carbonífero onde a concentração de CO2 era bem mais alta, ou seja, as emissões desse gás vão somente potencializar o efeito de crescimento das plantas se a humanidade não continuar com práticas abusivas de desmatamento.

Apesar de menos discutido o vapor d'água tem propriedades de efeito estufa que funcionam como uma retroalimentação ou também chamado de efeito \textit{feedback} \cite{Soden}. Este nome é dado por que quanto mais quente estiver, mais vapor d'agua pode ficar retido sem que tenha saturação ou precipitação na forma de chuva. Assim, nas ilhas de calor urbanas é possível a retenção de uma grande quantidade de água sem que haja chuva; e esse vapor d'agua vai potencializar a sensação térmica no ambiente devido a sua capacidade de efeito estufa.

Porém, mesmo que todo combustível fóssil seja consumido, a temperatura da Terra ainda estaria em um patamar agradável para a vida, como já ocorreu em outros períodos geológicos. A temperatura mínima e máxima que um planeta pode atingir depende principalmente da distância que este está da estrela em que orbita e o tipo dessa estrela. Portanto, Vênus sempre será muito quente para a vida como conhecemos e Marte não recebe energia solar o suficiente para sustentá-la - sem que se contrua alguma usina nuclear o que parece algo improvável em um futuro próximo. O foco de cientistas ambientais deve ser portanto em controlar os níveis de entropia da atmosfera: primeiro parando com o uso dos ar-condicionados e depois criando sistemas de refrigeração geotérmica.

Além disso, é um erro pensar que o aumento da temperatura na Terra vai causar um aumento no nível dos oceanos de forma direta. O aumento da temperatura irá influenciar o ciclo da água, formando mais tempestades que por sua vez poderão aumentar rios e lagos, que podem ser direcionadas para regiões áridas com o uso da refrigeração geotérmica.

\section{Ilhas de calor e Rio Grande do Sul}

As ilhas de calor formadas pelas metrópoles do Sudeste que possuem uma alta concentração de ar-condicionados fazem com que se tenha uma dificuldade na saturação do vapor d'agua. Em outras palavras a atmosfera na região das metrópoles acaba tendo uma grande concentração de água em dias quentes, que não precipita com facilidade devido ao uso intensivo de ar-condicionado e grande quantidade de queima de combustível fóssil - muitos automóveis em um pequeno espaço. Lembrando do que já foi explicado anteriormente que a água tem um efeito de feedback em relação ao efeito estufa, exceto em caso de dias muito nublados.

Essa massa de ar com alta saturação de água segue o fluxo natural do calor, ou seja, ela segue em direção ao pólo Sul onde encontra uma frente fria que força sua saturação e por isso gera os alagamentos e enchentes que estão sendo sentidos na Região Sul do Brasil. As catastrófes vistas na Região Sul podem voltar a ocorrer se não houver uma mudança na forma que a região Sudeste se refrigera. Porém, se a refrigeração geotérmica começar a ser usada em escala, além de evitar essas tragédias, poderemos ver efeitos positivos em relação ao combate da desertificação no sertão nordestino. Os problemas opostos entre o Nordeste e Sul do país estão ambos conectados à alta entropia atmosférica das regiões metropolitanas.