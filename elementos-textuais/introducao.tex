\chapter*[Introdução]{Introdução}
\addcontentsline{toc}{chapter}{Introdução}

As mudanças climáticas são um dos principais campos de estudos nas últimas décadas, causando grande preocupação dentro e fora do meio científico. A príncipio foi descoberta a sua relação com a queima de combustível fóssil que incrementa o Efeito Estufa em um processo que foi denominado \textit{Aquecimento Global}. Porém, logo percebeu-se que o planeta não estava aquecendo de forma uniforme; inclusive com invernos mais rigorosos em alguns locais. Por isso, o fenômeno foi renomenado para \textit{Mudanças Climáticas} ou \textit{Crise Climática}.
\par 
A disparidade das previsões de cientistas climáticos com a realidade tem sido percebida pela sociedade ao longo dos últimos anos, o que gera angústia social e paralisia; por não sabermos como melhor atuar para resolver o problema. Esta incerteza nas previsões metereológicas atrapalham na agricultura e em outros setores da economia. Além de já ter causado mortes e colocado vidas em risco devido à problemas como enchentes, alagamentos, estiagens e outros eventos catastróficos. 
\par 
Essa tese tem como por objetivo fornecer um melhor entendimento sobre a termodinâmica da atmosfera terrestre, bem como sugerir tecnologias para estabilizar o clima e mitigar possíveis eventos climáticos extremos do futuro. A princípio, o foco é no entendimento do equívoco em colocar a queima de combustíveis fósseis como protagonista nas catrástofes climáticas que têm ocorrido ao redor do mundo. Em seguida serão apresentadas as ferramentas para eliminar os eventos climáticos extremos e, por consequência, obter uma maior estabilidade climática. Por fim, serão apresentados possíveis trabalhos futuros e campos de estudo para aprofundar o conhecimento do funcionamento climático global.