\chapter[Conclusions]{Conclusions}

Geothermal refrigeration technology, when applied with modern agriculture techniques, has the potential to improve food yield and quality - helping to combat hunger that still plagues the world. Additionally, when applied in areas surrounding desertified regions, it can redirect water from melting polar ice caps to combat desertification. In summary, this thesis should not only be read as a hope against the Climate Crisis, but as the beginning of a Green Revolution on the planet. If the results of the experiment described here are positive, it could mark the start of global climate agreements that address both the fight against hunger and desertification worldwide.

The fossil fuils are essential for the functioning of modern society, and even though they are currently considered the cause of climate disasters, it is still difficult to imagine a society without their use. The main objective of this thesis is to place the blame for climate change where it belongs: on air conditioners. However, fossil fuels are still a non-renewable energy source, and society must begin to plan for the \textit{Post-Oil Era}.