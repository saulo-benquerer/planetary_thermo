\chapter[Greenhouse Effect]{Greenhouse Effect}

The Greenhouse Effect is the process of infrared rays reabsortion by the Earth's atmosphere. It is essential for life, because otherwise the planet would be much colder: around -18°C \cite{Nasa-cold}. The problem with the current way of studying this phenomenon is that we are trying to dissect the contribution of the various components of the atmosphere and study them separately.

What we call Greenhouse Effect is the resistance generated by the solar radiation pressure in opposition to the radiation reflected by the planet's surface. We know that the radiation emitted by a star decreases with the square of the distance; so the maximum temperature that Earth can reach depends much more on the intensity of solar radiation received than on the composition of the atmosphere itself. This is because a planet's position relative to its star not only affects the amount of energy received but also influences the rate at which it loses heat to outer space. Since there is no perfect vacuum, the temperature of outer space around a planet is also directly influenced by its distance from its star \cite{Cosmic-wave}.

In other words, even as the Greenhouse Effect affects a lot global climate, there is a limit to its effect. The Greenhouse Effect acts like a blanket, trapping heat and preventing it from escaping into space, but it does not create more heat.  There is a maximum temperature that Earth can reach, which is influenced by the position of the Earth relative to the Sun and the intensity of radiation sent by it. These two factors change only on cosmic time scales, so it is possible to stabilize the planet's climate in the long term.

The poles receive less solar radiation and, due to the ice also reflect some of it, enhancing the cooling effect of the area. This creates a constant flow of heat between the Equator and the poles. At all times, the planet reflects part of the heat as light energy into outer space. That is why when Voyager took its last photo, the planet could be seen as the "Pale Blue Dot", if Earth did not radiate energy to outer space all the time, our planet could not be seen (\autoref{blue-dot}).

\begin{figure}[ht]
    \centering
    \includegraphics[scale=0.25]{pictures/pale-blue-dot.jpg}
    \caption{Earth's photograph taken from Voyager 1}
    \label{blue-dot}
    \legend{Fonte: NASA/JPL-Caltech}
\end{figure}

This is obviously a simplified view of the issue, as ice intensifies the degree of reflection of solar rays and can reach a critical point that generates Ice Ages. Another reason why it does not make sense to dissect the degree of greenhouse effect of each atmospheric component is that depending on the concentration, its effect can reverse. Very cloudy days and/or with a lot of atmospheric smoke cause more solar rays to be reflected by the atmosphere than reabsorbed by the Greenhouse Effect, that is, the same components can affect the climate differently under various circumstances.


Besides, due to the rotation of the planet, there is a difference between day and night on opposite sides of the Earth, and this difference also influences terrestrial winds. During the night, the planet loses thermal energy more intensely to outer space, which ends up bringing winds from the warmer areas to these regions due to the natural flow of heat.

The most famous greenhouse gas is carbon dioxide, which is produced mainly by the burning of fossil fuels such as oil or coal. However, there are other gases that have this property, such as methane - which generates a lot of discussion in terms of the effect of livestock farming on climate change.
Plants absorb carbon dioxide to produce organic matter and many evolved during the Carboniferous period when CO2 concentration was much higher. For this reason, emissions of this gas will only enhance the growth effect of plants if humanity does not continue with abusive deforestation practices.

Dispite less discussed, water vapor has greenhouse properties that work as a feedback effect \cite{Soden}. This name was given because the warmer it is, the more water vapor can be retained without saturation or precipitation in the form of rain. Thus, in urban heat islands, it is possible to retain a large amount of water without rain; and this water vapor will enhance the thermal sensation in the environment due to its greenhouse effect capacity.

However, even if all fossil fuels were consumed, the Earth's temperature would still be at a pleasant level for life, as has occurred in other geological periods. The minimum and maximum temperature that a planet can reach depends mainly on the distance it is from the star it orbits and the type of star. Therefore, Venus will always be too hot for life as we know it, and Mars does not receive enough solar energy to sustain it - unless some nuclear power plant is built, which seems unlikely in the near future. The focus of environmental scientists should therefore be on controlling atmospheric entropy levels.irst by stopping the use of air conditioners and then by creating geothermal cooling systems.

Besides, it is a mistake to think that the increase in temperature on Earth will directly cause an increase in ocean levels. The increase in temperature will influence the water cycle forming more storms that can increase rivers and lakes, which can be directed to arid regions using geothermal cooling.

\section{Urban Heat Islands and Rio Grande do Sul Floods}

The urban heat islands formed by the cities of the Southeast that have a high concentration of air conditioners, making it difficult for water vapor to saturate. The atmosphere in the metropolitan areas ends up having a large concentration of water on hot days, which does not precipitate easily due to the intensive use of air conditioning and a large amount of fossil fuel burning - many cars in a small space. Remembering what has already been explained earlier that water has a feedback effect in relation to the greenhouse effect, except on very cloudy days.

The air with high water saturation follows the natural flow of heat, meaning it moves towards the South Pole where it encounters a cold front that forces its saturation, resulting in the floods and inundations being experienced in the Southern Region of Brazil. The disasters seen in the Southern Region can recur if there is no change in how the Southeast region cools itself. However, if geothermal cooling begins to be used on a large scale, not only could these tragedies be avoided, but we could also see positive effects regarding combating desertification in the northeastern hinterland. The opposing problems between the Northeast and South of the country are both connected to the high atmospheric entropy of metropolitan regions.